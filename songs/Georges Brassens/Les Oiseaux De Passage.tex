
\beginsong{Les Oiseaux De Passage}[by={Georges Brassens}]

 \beginverse
  \[N.C.]Ô vie heureuse \[Dm]des bourgeois ! Qu'avril bour\[C]geonne
  Ou que décembre \[Bb]gèle, ils sont fiers et con\[A]tents
  Ce pigeon est ai\[Dm]mé trois jours par sa pig\[C]eonne
  Ça lui suffi\[Bb]t, il sait que l'a\[A]mour n'a qu'un \[Dm]temps
 \endverse

 \beginverse
  Ce dindon a tou\[Dm]jours béni sa desti\[C]née
  Et quand vient le mom\[Bb]ent de mourir, il faut \[A]voir
  Cette jeune oie en \[Dm]pleurs : "C'est là que je suis \[C]née
  Je meurs près de ma \[Bb]mère et \[A]j'ai fait mon devo\[Dm]ir"
 \endverse

 \beginverse
  Elle a fait son devo\[Dm]ir c'est-à-dire que \[C]oncques
  Elle n'eut de souh\[Bb]ait impossible elle \[A]n'eut
  Aucun rêve de lune\[Dm], aucun désir de j\[C]onque
  L'emportant sans ra\[Bb]meurs sur \[A]un fleuve inconnu \[Dm]
 \endverse

 \beginverse
  Et tous sont ainsi \[Dm]faits, vivre la même v\[C]ie
  Toujours pour ces gens-\[Bb]là cela n'est point hi\[A]deux
  Ce canard n'a qu'un bec\[Dm] et n'eut jamais en\[C]vie
  Ou de n'en plus avo\[Bb]ir ou \[A]bien d'en avoir \[Dm]deux
 \endverse

 \beginverse
  Ils n'ont aucun besoi\[Dm]n de baiser sur les \[C]lèvres
  Et, loin des song\[Bb]es vains, loin des soucis \[A]cuisants
  Possèdent pour tout \[Dm]cœur un viscère sans \[C]fièvre
  Un coucou ré\[Bb]gulier et \[A]garanti dix an\[Dm]s
 \endverse

 \beginverse
  \[N.C.]Ô les gens bien heur\[Dm]eux !... Tout à coup, dans l'es\[C]pace
  Si haut qu'ils semblent \[Bb]aller lentement, un grand \[A]vol
  En forme de trian\[Dm]gle arrive, plane et \[C]passe
  Où vont-ils ? qui son\[Bb]t-ils ? Comme \[A]ils sont loin du s\[Dm]ol !
 \endverse

 \beginverse
  \[N.C.]Regardez les pa\[Dm]sser ! Eux ce sont les sauva\[C]ges
  Ils vont où leur dés\[Bb]ir le veut : Par-dessus \[A]monts
  Et bois, et mers, et ve\[Dm]nts, et loin des escla\[C]vages
  L'air qu'ils boivent \[Bb]ferait écla\[A]ter vos poumon\[Dm]s
 \endverse

 \beginverse
  Regardez-les ! Av\[Dm]ant d'atteindre sa chi\[C]mère
  Plus d'un, l'aile rom\[Bb]pue et du sang plein les ye\[A]ux
  Mourra. Ces pauvres \[Dm]gens ont aussi femme et m\[C]ère
  Et savent les ai\[Bb]mer aussi \[A]bien que vous, \[Dm]mieux
 \endverse

 \beginverse
  Pour choyer cette \[Dm]femme et nourrir cette \[C]mère
  Ils pouvaient deven\[Bb]ir volailles comme vou\[A]s
  Mais ils sont avant tout\[Dm], des fils de la chi\[C]mère
  Des assoiffés d'a\[Bb]zur, des \[A]poètes, des f\[Dm]ous
 \endverse

 \beginverse
  Regardez-les, vieux co\[Dm]qs, jeune oie édifian\[C]te !
  Rien de vous ne pour\[Bb]ra monter aussi haut qu'eu\[A]x
  Et le peu qui viendra\[Dm] d'eux à vous, c'est leur \[C]fiente
  Les bourgeois sont \[Bb]troublés de \[A]voir passer les gue\[Dm]ux \rep{2}
 \endverse

\endsong
